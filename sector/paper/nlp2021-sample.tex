%#!platex
\documentclass[
  platex, dvipdfmx,  % ワークフローは必ず明示的に指定する
]{nlp2021}
%#!uplatex
%\documentclass[uplatex,dvipdfmx]{nlp2021}
%#!lualatex
%\documentclass[lualatex]{nlp2021}


% パッケージ
\usepackage{xcolor}  %
\usepackage{graphicx}  % グラフィックス関連
\usepackage{pxrubrica}        % ルビ
\usepackage{url}
\usepackage[square,numbers]{natbib} % 参考文献のフォントサイズを変更する用


%% option 不要な場合はコメントアウト
\usepackage{bxjalipsum}       % ダミーテキスト
\usepackage{hyperref}
\usepackage{pxjahyper}
\hypersetup{
	colorlinks=true, 
    citecolor=blue, 
    linkcolor=blue,
	pdfborder={0 0 0},
}
\def\bibfont{\small} % 参考文献のフォントサイズを指定


% 著者用マクロをここに入れる
\newcommand{\pkg}[1]{\textsf{#1}}
\newcommand{\code}[1]{\texttt{#1}}
\newcommand{\comment}[1]{\textcolor{red}{#1}}
%%%%%%%

\title{NLP2021論文投稿用スタイルガイドおよびサンプル原稿}
\author{%
  言語太郎 \\ 言語大学処理学部 \\ \texttt{taro@example.com}\and
  言語花子 \\ 言語大学処理学部 \\ \texttt{hanako@example.com}}

\begin{document}

\maketitle

\section{はじめに}
この文書は、言語処理学会年次大会への投稿論文を作成する際のインストラクションである.
NLP2021より、賞選考コストの削減の観点から、投稿論文のフォーマットを規定する.
そのため、規定のフォーマットを満たす年次大会論文投稿用スタイルファイル(\code{nlp2021.cls})を配布する.
この文書自体が当該年次大会論文投稿用スタイルファイルを用いて作成されている.
よって、この文書を参考に投稿論文の原稿を作成することを推奨する.


\subsection{基本設計}
LaTeX版NLP2021文書クラスはW3Cにより策定されている『日本語組版の要件』\cite{JLREQ}に準拠することを目指す\pkg{jlreq}クラスをベースにしている.
ただし、本文書クラスでは紙面スペースの都合上、余白値をかなり詰めるように設定している.
例えば、行間は\ruby{外国人参政権}{がい|こく|じん|さん|せい|けん}のようにルビを振れる最小限の余白に設定してある.

NLP分野の論文では、単純なテキストのみならず、しばしば数式
%
\begin{equation}
P(B\mid A) = \frac{P(A\mid B)P(B)}{P(A)}
\end{equation}
%
や箇条書き
%
\begin{itemize}
\item 第1の項目
\item 第2の項目
\end{itemize}
%
といった構造も用いられるが、LaTeX版ではこれらもよく知られた文書クラス(例えば\pkg{jsarticle}等)と同様のシンタックスで利用できる.

本文書クラスの仕様の詳細については\code{README-latex.md}を参照されたい.


\subsection{Word版テンプレート}
Word版テンプレートは、前述のLaTeX用に定義されたNLP2021文書クラスに準拠して作成されている.
Word版でも、数式や箇条書きなどはWord上の機能を用いて挿入することができる.

LaTeX版文書クラスでの禁止事項およびWord版で投稿される論文が満たすべき規定については、\ref{sec:contents-format1}節および\ref{sec:contents-format2}節に詳述する.


\subsection{クレジット}
LaTeX版のクラスファイル(\code{nlp2021.cls})は、東京大学宮尾研究室 朝倉卓人氏のご厚意により年次大会用に提供していただいた.

また、Word版のテンプレートはLaTeX版のフォーマットに従って理化学研究所 吉野幸一郎氏により作成していただいた.




\section{投稿論文の必須要件}
\label{sec:contents-format1}

投稿論文に関する規定には、必ず満たさなければいけない「必須要件」と、賞選考のために満たすことを前提とする要件の2種類がある.
本節では、必ず満たす必要のある「必須要件」について述べる.

\begin{enumerate}
\item 原稿は本文は4ページ以内、本文と参考文献を含めて5ページ以内、付録は独立した1ページ以内
\item 各ページの余白は上下3cm、左右2cm以上
\end{enumerate}

1に関しては、本文と参考文献を合わせて最大で5ページの原稿を投稿することができるが、参考文献がどんなに少量であったとしても、5ページ目に本文が入ってはいけないことを意味する.
また、本文および参考文献とは別に、著者が望む場合は付録(Appendix)\footnote{付録に関しては、\ref{sec:appendix}節を参照のこと.}を1ページ分、原稿につけることができる.つまり、最大で6ページ分の原稿を投稿することができる.




2に関しては、投稿論文に含まれる全てのページに対して余白の規定を満たす必要がある(付録も含む).

本節記載の1および2の要件を満たしていない場合は、発表取り消しとなる可能性がある.
投稿時には十分に気をつけて投稿すること.
軽微かつ容易に修正可能な場合は、プログラム委員会で予告なく原稿を修正する可能性がある(その場合は発表取り消しにはならない).



\section{投稿論文の体裁}
\label{sec:contents-format2}
\ref{sec:contents-format1}節冒頭で述べた通り、論文の体裁に関する規定には、必ず満たさなければいけない「必須要件」と、賞選考のために満たすことを前提とする要件の2種類がある.
本節では、「賞選考のために満たすことを前提とする要件」を述べる.

賞選考コストの削減の観点から、投稿論文のフォーマットを規定する.
その詳細を本節に記載する.
規定フォーマットに明らかに従っていない場合は、\textbf{予告なく優秀賞・若手奨励賞などの一部の賞の選考過程から除外されることがある.}

ただし、賞選考のコスト削減の観点から生じた施策なので、本節に記す規定フォーマットを満たさない原稿であっても、前節の必須要件が満たされていれば\textbf{発表自体が取り消されることはない}.



\subsection{本文}

\paragraph{LaTeX版}
文書クラスが定義する以下についての変更は禁止とする.
\begin{itemize}
\item 用紙サイズ
\item フォントサイズ
\item 欧文フォント (利用するフォントによって文字数に異なりが生じるため)
\item 余白の大きさ
\item 行間、行数、文字数(特に\code{baselinestretch}を変更しないこと)
\end{itemize}


\paragraph{Word版}
LaTeX版で定義された文書クラスと同等のテンプレートを実現するため以下のような定義を行っている.
これらを変更することは禁止とする.
\begin{itemize}
    \item 用紙サイズはA4、組版は2段組とする.
    \item フォントサイズは以下のように定める.
    \begin{itemize}
        \item 論文表題: 16pt
        \item 著者名: 10-11pt
        \item 大見出し: 14pt
        \item 中見出し: 12pt
        \item 小見出し: 11pt
        \item 本文: 10pt
        \item その他本文中の数式などの文字: 10pt
        \item 図表等のキャプション: 10pt
        \item 上記以外のクラス、例えばアルゴリズムなどを記述する場合の文字: 10pt以上
    \end{itemize}
    \item 論文の余白は以下の通り定める.
    \begin{itemize}
        \item 上下: 3cm
        \item 左右: 2cm
    \end{itemize}
    \item 行数は45行, 各行の文字数は全角23文字
    \item ルビを振る場合、行間を固定値とし、その値を14.9ptとする.設定する場合、「段落」$\xrightarrow{}$「インデントと行間の変更」$\xrightarrow{}$「行間」から指定する.
\end{itemize}


\subsection{Write in English}
This paragraph shows an English sample.
There is no problem to prepare your manuscript in English.
If you write on LaTeX, please use the distributed style file.
Any changes on the style file (\code{.cls}) are prohibited.
If you write on Microsoft Word, please use the distributed sample file without changing its layout.
Using ``Times New Roman'' is suggested.

なお英語での原稿作成について、LaTeX版の場合は配布するスタイルファイルを用いて記載すれば問題ない.Word版の場合は配布テンプレートを用いて、レイアウト等については変更しないこと.本文は、スタイルファイルで規定される通りTimes New Romanで記載のこと.



\subsection{図、表、例文等}
図、表、例文等、本文とは独立に表記される領域における文字サイズも、基本的には本文と同じ10ptを推奨する.

ただし、図や例文などは、別のツールで作成したオブジェクトを原稿に埋め込むことなどを鑑みると、中の文字の正確なサイズを知るのは難しいので図中のフォントサイズは規定しない(10pt以下の文字サイズがあっても規定違反とはしない).
ただし、A4印刷で読める大きさは担保するように留意すること.

表に関しても、情報を多く記載する必要性がある場合を鑑み、\verb|\small| (9pt)相当のフォントサイズまでは必要であれば利用しても良いこととする.
また、\verb|\tabcolsep|などを使って各セルの横方向を詰めることは許容する.
ただし、詰めすぎて読みにくくならないように留意すること.





\subsection{参考文献}
本文の直後に、\textbf{参考文献}のセクションを設け、本文の中で参照した参考文献の詳細を列挙すること.
参考文献の体裁については厳密に指定はしない.
参考文献の並び順は、本文での参照順や著者のアルファベット順など、どのような順番でも良いこととする.

必要に応じて独自の参考文献のスタイルを用いることができる.
年次大会の推奨設定は以下とする.
\begin{quote}
\verb|\bibliographystyle{junsrt}|\\
\verb|\bibliography{j_yourrefs}|
\end{quote}
Word版では「参考資料$\xrightarrow{}$引用文献の挿入」を利用することを推奨する.引用の方法は、ISO 690: 参照番号を利用する.
ただし、適切に番号の対応が取られていればWord版引用文献の機能を利用することは必須ではない.

つまり参照は[1]や[2, 3]といった数字で本文内で参照され、その数字に合わせて参考文献が記載される.
LaTeX版の本文で参考文献を参照する際には、
\verb|\cite{Article_01}|
といった形式で参照する.
%
著者の名前は、略記はせずにフルネームを記載することを推奨する.

以下、参照の参考例である.
\begin{itemize}
\item 論文誌の参照例 \cite{Article_01}
\item 本の参照例 \cite{Book_02}
\item 国際会議の参照例 \cite{Inproc_03}
\item 技術報告の参照例 \cite{Techrep_05}
\item Webページの参照例 \cite{Web_06}
\end{itemize}

また、参考文献が1ページに入りきらない場合など、参考文献のセクションに限り、フォントサイズを小さくすることを認める.
ただし、1ページに入る場合には極力フォントサイズを変更しないことを推奨する.

\subsection{脚注}
補足情報を入れるために脚注(\code{footnote})を利用することができる.\footnote{脚注の例である.}
脚注はページの下部に9ptで表記する.
また、脚注は論文全体で1から番号をつけ、閉じ括弧などの記号を伴って、どの脚注がどこに対応するか明確にわかるようにする.
脚注は本文と水平線(横線)で分割される.\footnote{ツールを参照する際に脚注にURLのみで参照する事例が散見されるが、ツールに紐づく文献などを積極的に参考文献にして追加することを推奨する.}
なお、Word版においては「参考資料$\xrightarrow{}$脚注の挿入」から脚注を利用することができるが、本テンプレートが利用している通り、脚注箇所を明確にするためアラビア数字以外の文字を脚注記号として利用することを推奨する.



\subsection{付録(Appendix)}
\label{sec:appendix}
本文とは別に付録(Appendix)を1ページつけることができる.
付録は、追加の実験結果や詳細な実験設定、式の証明などを著者が記載したい場合に利用することを想定しており、基本的には付録をつける必要はない.

付録に関しては、本サンプルで利用している年次大会指定のフォーマットに従う必要はない.
ただし、必須要件に入っている上下左右の余白に関しては規定を満たす必要がある.
本文領域に関しては、どのような形式で付録を作成するかは著者の裁量による.

付録に記載の内容は、賞選考時には考慮されない.
つまり、賞選考の審査員は賞選考時に付録を読まないことを前提としている.よって、本文から付録を参照する際には、その参照がなくても本文内で議論が完結するような書き方が必要である.逆に付録の情報に基づいた議論が本文中であったとしたら、それは審査で不利に判断される可能性がある.



投稿時には、本文および参考文献に続けて付録を配置し、単一のPDFとして投稿する必要がある.
また、付録は付録だけで独立した1ページで構成する.
つまり、本文や参考文献のページ数が上限に達していなくても、付録は独立した1ページが上限となる.
単一の原稿として作成している場合は、付録の直前で必ず改ページをおこない、本文や参考文献とは独立したページとなるように注意する.







\section{参考情報}
本節には、その他、原稿執筆に有益と考えられる情報を記す.

\subsection{図の挿入}
%
\begin{figure}[t]
\centering
\includegraphics[width=3cm]{example-image-a}
\caption{何らかの図}
\label{fig:sample}
\end{figure}

図のキャプションは図の下につける.
図\ref{fig:sample}は実際の挿入例である.


\paragraph{LaTeX版}
図の挿入は通常\pkg{graphicx}パッケージによって行う(図\ref{fig:sample}参照).
クラスオプションにワークフロー(\code{dvipdfmx}等)を指定していれば、各パッケージを読み込む際に何度も同じオプションを指定する必要はない.


\paragraph{Word版}
図の挿入は挿入$\xrightarrow{}$図の機能によって行う.
図を挿入する場合、挿入した図を選択した際に表示される「図ツール」の「文字列の折り返し」から、「上下」を利用する.
また、「参考資料」から「図表番号の挿入」を選択し、図表番号と同時にキャプションを付与する.



\subsection{表の挿入}
図とは異なりキャプションは表本体の上に付ける.
表\ref{tab:sample1}は実際の挿入例である.
表\ref{tab:sample2}は表\ref{tab:sample1}のフォントサイズを\verb|\small| (9pt)に変更した例である.



\paragraph{LaTeX版}
表は\verb|\begin{table}...\end{table}|環境を使う.

\paragraph{Word版}
表組みもWordの「挿入」から表を追加できる.
また、図と同様に「参考資料」から「図表番号の挿入」を選択し、図表番号と同時にキャプションを付与する.
なお、Word版においてはフォントサイズを9ptとしてもあまり大きく余白を詰めることはできない.

\begin{table}[t]
\centering
\caption{適当な表}
\label{tab:sample1}
\begin{tabular}{llcc}
\hline
日本語 & Japanese & ほげほげ & ふげふげ\\
英語 & English & hogehoge & fugefuge\\
\hline
\end{tabular}
\end{table}
%
\begin{table}[t]
\centering
\small
\tabcolsep 3pt
\caption{適当な表(small バージョン)}
\label{tab:sample2}
\begin{tabular}{llcccccc}
\hline
\      &      &\multicolumn{3}{c}{データ1}&\multicolumn{3}{c}{データ2}\\
\      & 設定 & Pre. & Rec. &F1 & Pre. & Rec. &F1\\
\hline
Model1 & config1 & 23.04 & 30.11 &  25.6 & 23.04 & 30.11 &  25.60\\
Model2 & config1 & 23.04 & 30.11 & 23.04 & 23.04 & 30.11 & 23.04 \\
\hline
\end{tabular}
\end{table}

\subsection{色つけ}
投稿論文の原稿への色つけに関しては特に規定を設けない.
図表、本文も含めて、読者がより理解しやすいと著者が判断するのであれば、著者の裁量で自由におこなってよい.

\subsection{hyperref}
論文内の参考文献、式、セクション等へのハイパーリンクを埋め込みたい場合は、著者の裁量で自由におこなうことができる.


\subsection{Overleaf}
原稿執筆時にOverleafを利用して作成する人が多いと思われるが、特定のツールの使い方を年次大会で公式にサポートはしない.
ただし、利用時のTIPSとして、年次大会配布のファイルを置いただけでは日本語環境が整っていないという意味でコンパイルできない場合がある.
その場合は、latexmkrcを用意し、そこに日本語用の設定を記載する.
詳細はインターネットで検索すれば多くの情報を見つけられるので、そちらに譲る.
コンパイラはLaTeXを選択する.

\section{ダミーテキスト}
本節は以下ダミーテキストである.

\subsection{サブセクション1}
このサブセクションはダミーテキストである.

\subsubsection{サブサブセクション1}

\paragraph{パラグラフ}
ダミーテキストである.

\section{おわりに}
投稿論文に関する規定には、必ず満たさなければいけない必須要件(\ref{sec:contents-format1}
節)と、賞選考のために満たすことを前提とする要件(\ref{sec:contents-format2}
節)の2種類がある.

必須要件は、論文のページ数と余白に関する規定である.必須要件を満たさない論文は発表取り消しの場合もある.
一方、賞選考のために満たすことを前提とする要件は、賞選考コスト削減が主な理由であり、満たされていなくても発表が取り消されることはない.
ただし、優秀賞・若手奨励賞などの一部の賞の選考過程から除外されることがある.

年次大会論文投稿用スタイルファイルを使った執筆がどうしても自己解決できない場合は、プログラム委員会まで問い合わせること.

%%%%  ここまでが本文 4ページ以内

% 参考文献
\bibliographystyle{junsrt}
\bibliography{j_yourrefs} % ファイル名は適宜自分のbibファイルに置き換える

%%%%  ここまでが本文+参考文献 5ページ以内


% 付録(Appendix)
% 付録を付けない場合は、以下\end{document}以外を全てをコメントアウトする.
% 本文、参考文献に続けて作成する場合は、必ず \clearpage して新たなページとする
\clearpage
% 付録は別ツールで作成して、後で本文PDFに追加する方式でもよい
\appendix
% ここ以降はフォーマットを自由に変更可能
\onecolumn % onecolumnにしたい時の例
付録のサンプル
\section{付録}
\tiny
\jalipsum[20-23]{wagahai}

\end{document}
